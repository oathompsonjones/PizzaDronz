% Define the type of document.
\documentclass[11pt, a4paper]{article}

% Set a more sensible document size.
\usepackage[margin=0.75in]{geometry}

% Set the fonts.
\usepackage{fontspec}
    \setmainfont{Lato}
    \setmonofont[Contextuals=Alternate, Scale=0.8]{Fira Code}
    \makeatletter\renewcommand*\verbatim@nolig@list{}\makeatother

% Set the language.
\usepackage[british]{babel}
    \usepackage{csquotes}

% Prevent words from being split accross lines.
\tolerance=1\emergencystretch=\maxdimen\hyphenpenalty=10000\hbadness=10000

% Set up the bibliography.
\usepackage[sorting=none, style=science]{biblatex}
    \addbibresource{references.bib}
\usepackage[colorlinks=true, allcolors=blue]{hyperref}

% Allow syntax highlighting for code segments.
\usepackage[outputdir=build]{minted}
    \setminted{
        style=one-dark,
        linenos,
        autogobble,
        breaklines,
        breakautoindent,
        breakanywhere,
        mathescape,
        resetmargins,
        tabsize=4
    }

% Allow more advanced maths.
\usepackage{amsmath}

% Create better tables.
\usepackage{xltabular}
    \renewcommand{\arraystretch}{2}

% Allows images to be included.
\usepackage{graphicx, animate}
    \graphicspath{ {./images/} }

% Allow for drawing complex diagrams.
\usepackage{tikz}
    \usetikzlibrary{positioning}

% Allow disabling of floating objects.
\usepackage{float}

% Allows the use of subfiles.
\usepackage{subfiles}

% Removes numbers from headings and the contents page.
\setcounter{secnumdepth}{0}

% Sets the title.
\title{PizzaDronz Test Plan}
\author{Ollie Jones — s2153980}
\date{}

\newcommand{\baseURL}{https://github.com/oathompsonjones/PizzaDronz}

% Begin the document.
\begin{document}
\maketitle

\section{Requirements}
    The requirements for the PizzaDronz project are outlined in the \href{\baseURL/blob/main/Project%20Proposal.pdf}{project proposal}.
    In summary, the requirements are as follows:
    \begin{itemize}
        \item The system must be able to retrieve data from the REST API, including orders, restaurants and no-fly-zones.
        \item The system must be able to deserialise the JSON data returned by the REST API.
        \item The system must be able to validate each order to check if it should be delivered or not.
        \item The system must be able to calculate the optimal route from a given restaurant to Appleton Tower, without entering any no-fly-zones.
        \item The system must be able to stitch each route together to create a single route for all orders.
        \item The system must be able to serialise the route and order validation results into JSON and GeoJSON formats.
        \item The system must be able to provide these output in three files in the resultfiles directory.
        \item The system should be able to complete the entire process in under 60 seconds.
        \item The system should be data driven and scalable to allow for further development in the future.
        \item The system should be well documented to allow others developers to pick up the project in the future.
    \end{itemize}

\end{document}
